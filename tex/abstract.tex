\begin{abstract}
  Programming languages evolve over time, with new versions being published and new features being introduced, as well as older features being deprecated and discontinued. Existing software that isn't upgraded to use newer versions of its programming language usually contain idioms and workarounds for problems that are already solved, and the process of upgrading to implement these newer versions is called Source Code Rejuvenation. This concept is an important element in software maintenance and development, yet there is still a small amount of research regarding software engineering that focuses on it. This document aims to study how the developers at Boost discussed and dealt with the topic throughout its mailing list history.
\end{abstract}
